\section{Introduction}

THIS INTRODUCTION IS TAKEN FROM THE DSSG PAPER.  IT NEEDS TO BE
REWRITTEN.

For the past few decades, both the field of computing in America and
computer science in higher education have suffered from significant
underrepresentation of women and domestic students of color.  A
wide variety of projects have attempted to address this issue.
However, even after all of these efforts, a recent Taulbee survey
still suggests that only 17.9\% of bachelor's degrees in computer
science are awarded to students identified as female, 3.1\% to
students identified as Black or African-American, and 7.5\% to
students identified as Hispanic \cite{Taulbee2016}.  More work is
needed.

Many factors are at play in this problematic situation.  One that
is repeatedly identified is a ``pipeline problem'', with members
of these groups deciding early on that they do not belong in computer
science \cite{Gurer2002}.  As a result, the broader compute science
community has tried a variety of outreach activities
\cite{McGill2015,Decker2016}.

Over the past three years, our research team has explored the effect
of summer code camps on student interest and self efficacy.  Rather
than employing the more common approaches to computing the emphasize
playful aspects of computing (e.g., through robots, games, or
Minecraft) \cite{code-camp-survey-sigcse-2017}, we emphasized
meaningful uses of computing \cite{arts-coding,dssg-sigcse-2018},
which we hypothesize will help build students' self efficacy and
interest, particularly among students from groups traditionally
underrepresented in computing.

In this paper, we describe a very different approach to introducing
computing, one that emphasizes the digital humanities, particularly
problems related to language generation or analysis.  ADD MORE STUFF.

In section 2 of the paper, we describe the local context of the
camp, particularly the characteristics of our community and the
students who enrolled.  In section 3, we describe the design and
structure of the camp and provide additional detail about our
approach to the digital humanities.  In section 4, we provide a
broader view of the curriculum of the camp.  In section 5, we review
data from camper surveys.  Finally, in sections 6 and 7, we further
consider the implications of the data and our observations from the
camp.

\section{Background}

\subsection{Local context}

THIS SECTION IS COPIED FROM THE DSSG PAPER.  IT NEEDS SOME UPDATE.
OR MAYBE IT NEEDS DROPPING

Our institution is in a rural community (approximately 10,000
residents) in a relatively white state---91.4\% of the population was
identified as white in the latest census.  Our community has a large
population of working poor: While the poverty rate in our county
is over 30\%, the unemployment rate is under 3\%.  Surrounding
communities have similar populations.

In designing the camp, we strove to address the particular concerns
of our community.  While we wanted to address the underrepresentation
of girls and students of color, we also wanted to make sure that
the camp was accessible to students of lower socio-economic status.
Building these students' self-efficacy seems particulary important
as the opportunities presented by computing careers can make a huge
difference in their lives.  We used a variety of approaches to
support these students, including free tuition for students on free
or reduced lunch programs\footnote{Unfortunately, because many feel
a stigma in our schools, many students who would qualify for these
programs choose not to enroll.  The problem is most significant at
the high-school level, but it is clearly also in play at the
middle-school level.  To help address that stigma, we relied only
on families' self reports on the application form, and not on data
from the school district.}, providing meals and snacks for every
camper, subsidizing costs for all campers, and providing no-cost
options for early drop-off and late pick-up.

Because there are few educational enrichment activities available
to rural students, we drew students not only from our own community,
but also from neighboring communities.  Some came from as far as
sixty miles away, suggesting that there are far too few such
opportunities for students in rural areas.

Although we had originally capped the camp at 32 students, experience
from prior camps, which suggested that we'd have a bit of melt the
first day (one of the disadvantages of a low enrollment cost, we
expect) we allowed 38 students to enroll.  We had approximately 34
students at each day of camp.  N of these had previously taken a
separate camp on data science and social good, N in summer 2017
and N in summer 2018.

Both college students and local high-school students served as
counselors for the camps.  We generally tried to maintain a ratio
of about one counselor per four campers in the classroom; that ratio
varied slightly as counselors took breaks or worked on preparation
for other activities.  As a result of this relativley low ratio,
counselors were able to provide support promptly and effectively.
And, because counselors were not pressed for time, they were able
to encourage campers to find answers to their problems themselves,
rather than provide immediate solutions.

\subsection{Digital Humanities}

ADD DEFINITION AND NOTES.

DESCRIBE ERIK'S COURSE.  For example, one of the first digital humanities
courses at our institution, which focused on literary studies, took a
three-part approach.  Although students began the semester with a 
environmental scan of the digital humanities, identifying everything
from markup languages for poetry to issues surrounding the dominance
of American and Western Europe in digital humanities work, the course
focused on three primary issues: writing, map-based analysis, and the
use of coding as a tool to explore texts.

\section{Designing the camp}

ADD A FEW SENTENCES

\subsection{Primary activities}

Our goal was to draw inspiration from the digital humanities

\subsubsection{Introduction to HTML and Web pages}

Introduces formal language.  Provides fun activity.

\subsubsection{Choose your own adventure}

Undertanding of hypertext and linking.

\subsubsection{Language generation}

Contextualize with Twitter 'bots.  Helps them think about
the structure of language.  Tends to be amusing.  Also provides
some insight into language.

\subsubsection{Language analysis}

Particularly of text files and tweets.  Gives understanding of poer
of computers.  Begins to involve a broader range of algorithmic
issues.

\subsubsection{MadLibs}

Provides context for variables.  Familiar domain.  Also fun.

\subsection{Platforms and languages}

In every other camp, we have started with a block programming language,
such as Scratch.  Because a significant portion of this camp had
either participated in a prior camp or had encountered block
programming in school, we looked for languages that would feel less
familiar, but would still be appropriate for this level of students.

We settled on Racket.  Interpreted.  Built in Web-server with natural
Web page model.  Could also tell students

MORE TO COME

\subsection{Final projects}

Conclude with open-ended project: "With your partner, do something
interesting with what you've learned.  Present it to the class."

Brainstorming, presentation, and partnering.

16 projects.  Two basic Madlibs.  Two
more advanced Madlibs that including one that combined ideas from Madlibs 
and hypertexts; in one case, users would fill out information ...

\subsection{Structuring topics and days}

THIS SECTION IS COPIED VERBATIM FROM THE DSSG PAPER.  IT NEEDS
REWRITING.

In designing the camp, we considered both technical outcomes---core
computational techniques like conditionals, loops, and subroutines--as
well as more conceptual outcomes---we wanted students to feel the
power that comes with writing programs and to realize that they had
the ability to write such programs.  We had also identified a
necessary set of skills for students to complete the final projects
in data science.  As is the norm in designing such camps, we then
considered the natural flow of ideas, reflecting on the transition
from block-based languages to a text-based language and an appropriate
spiral approach that helped students revisit topics in different
levels of depth.

While our primary focus was on computing topics, we knew from past
experience in this camp and others that students of this age group
would not learn well if they spent the day in front of the computer.
Hence, the camp schedule incorporated frequent ``away from computer''
activities including short breaks to play active and engaging games.
Snacks were offered during one of the breaks before lunch, and
another after lunch.

Moving the activities away from the computer did not mean that we
necessarily moved them away from thinking about core topics.  For
example, one the first day of camp we moved to an outdoor amphitheater
to carry out the algorithms they had developed for reuniting families
after a disaster.  We also included CS Unplugged \cite{csunplugged}
activities to introduce Computer Science concepts in a fun and
active environment.  During afternoon breaks, we invited faculty
and students in the Computer Science department to introduce campers
to their research.  

We also exposed students to the broader view of ds4sg early and
repeatedly.  Most days, counselors presented a case study that
integrated new concepts with social good data. For example, our
second case study taught campers how to organize data about weather
changes in a way that would allow them to analyze it in Python.
Most of the last day was devoted to working on final projects,
getting them ready to present to their friends and family.

\section{Camp structure}

NEED TO INSERT A CHART OR SOMETHING SIMILAR SHOWING THE OVERVIEW OF THE
CAMP.

\section{Assessment}

In order to measure the campers' change in self-efficacy and coding
confidence in a quantitative manner, we used pre- and post-survey
instruments based on upon the Georgia Computes! instruments
\cite{Bruckman2009} that ask students to respond to various statements,
such as ``I look like a computer scientist'' and ``I like the
challenge of computing'' on a seven-point Likert scale.  The questions
appear in fig.~1.  Through these surveys, we were able to get a
good sense of what subjects and concepts the campers were comfortable
with prior to attending the camp, so as to measure any significant
changes in their outlook by the end of the camp.  

NEED TO INSERT THE REVISED SURVEY.

\begin{figure}
{\small
1:\textit{I look like a computer scientist.} \\
2:\textit{Boys can do computing.} \\
3:\textit{Girls can do computing.} \\
4:\textit{I know a lot about computing.} \\
5:\textit{I can become good at computing.} \\
6:\textit{I know more than my friends about computers.} \\
7:\textit{I like the challenge of computing.} \\
8:\textit{Computer science is cool.} \\
9:\textit{I feel comfortable using a computer.} \\
10:\textit{I want to have a job in computing.} \\
11:\textit{I want to learn more about computing.} \\
12:\textit{Computer scientists are creative.} \\
13:\textit{Solving problems is fun.} \\
14:\textit{Computing is collaborative.} \\
15:\textit{I like computing.} 
}
\caption{Survey statements}
\end{figure}

DATA?  A CHART ISN'T NECESSARY, BUT SOME STATEMENT ABOUT WHAT IS AND
IS NOT SIGNIFICANT.

\section{Discussion}

Informal obervations

A student who had been reluctant to work with others or ask for help
in a prior camp was enthusiastic about the project and showed great
flair in their presentation.

During one of the scavenger hunts, one of the campers said "I never
thought about attending college.  But this place is cool.  Do you
think I could get in?"

Attracted fewer girls than our camps focused on social good.  We had
expected that a focus on language would also broaden participation.  

Packed in too much.

A visiting K-12 teacher noted that even some of the exercises helped
students think more deeply about language, particularly Madlibs and language
generation.  It could be that this is a natural way to incorporate more
computing in language courses not so much to teach computing, but to 
help students gain a deeper understanding of language.

Parents called and emailed after the camp asking for access to the
kid's projects.

Students appreciated knowing that (some of) their work was on the
WWW and could be accessed from anywhere.  Empowering!

\section{Conclusions}

WRITE ABOUT STRENGTHS AND POTENTIALS.

Although many middle-school outreach activities focus on students'
perceived interests, such as video games or ``exciting'' technologies,
it is clear from our experiences that computing for social good
provides a platform for helping students grow in self-efficacy and
interest.  We hope that our modest project will inspire others to
take a similar approach.

We have also found that while students can be intimidated by a text-based
programming language and a professional programming environment, they also
feel empowered and gain additional self efficacy in using such an environment.
While we are less comfortable recommending that others regularly take similar
approaches, our informal conversations with the campers suggest that the
approach is worth investigating further
