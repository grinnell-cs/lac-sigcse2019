\begin{abstract}

Over the past decade, politicians, leaders, and pundits have called for computing and computer science education opportunities to be made available earlier and earlier. Such calls have led to the creation of a wide variety of offerings for students at middle-school and even elementary levels, including summer "code camps" targeted at middle-school students. Such camps often emphasize "fun" aspects of computing, such as games and robots [20]. In contrast, research at the collegiate level suggests that meaningful applications of computing, such as computing for social good, are more successful and building and sustaining interest, particularly among students from groups traditionally underrepresented computing.

In this project, we developed and offered a summer camp that draws upon ideas and approaches from the digital humanities. The digital humanities, or DH, explores the use of algorithms and computation in support of broader humanistic inquiry. Because DH reveals different ways to apply algorithmic and computational thinking, DH has the potential to attract students who might not otherwise consider computing.

In this paper, we introduce central issues in the digital humanities, explain the rationale for the camp design, describe the camp curriculum, and reflect on successful and less successful aspects of the camp. Among other things, we consider how to introduce digital humanities topics to students who have not yet heard of the humanities and explore the utility of such topics for this age group. We also present preliminary data on the effects of the camp on students' self-efficacy and interest in computing.

\end{abstract}
